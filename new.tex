\documentclass{report}
\usepackage[margin=18mm]{geometry}
\parindent = 0pt
\usepackage[svgnames]{xcolor}

\usepackage{klrw_diagrams2}

\usepackage{cmap} % fix search and cut-and-paste in Acrobat
\usepackage[T1]{fontenc}
\usepackage[utf8]{inputenc}
\setcounter{secnumdepth}{0}

\usepackage[skins,listings]{tcolorbox}

\DeclareTotalTCBox\keyword{ O{} v }{
  fontupper=\sffamily,
  nobeforeafter,
  skin=tile,
  verbatim,
  on line,
  tcbox raise base,
  top=0pt,bottom=0pt,left=0mm,right=0mm,
  colback=OldLace,
  colupper=ForestGreen,
  #1}
{#2}

\lstdefinestyle{tikz}{style=tcblatex,
  classoffset=0,
  texcsstyle=*\color{blue},%
  deletetexcs={begin,end},
  moretexcs={,%
    pgfdeclarehorizontalshading,
    pgfuseshading,
    node,
    useasboundingbox,
    draw
  },%
  classoffset=1,
  keywordstyle=\color{MediumBlue},%
  morekeywords={tikzpicture,shade,fill,draw,path,node,child,line,width,rectangle},
  classoffset=2,
  keywordstyle=\color{ForestGreen},%
  morekeywords={solid ghost, solid,dot,affine,red,ghost, string, double}
}

\DeclareTCBListing{tikzcode}{ !O{} }{%
  skin=bicolor,
  colframe=Sienna,
  colbacklower=Lavender,
  colback=OldLace,
  lefthand width=40mm,
  listing style=tikz,
  sidebyside,
  sidebyside gap=4mm,
  text and listing,
  text outside listing,
  boxsep = 0pt,
  #1
}

\makeatletter
\author{Andrew Mathas}
\usepackage{tikz}
\usetikzlibrary{shadows.blur}
\tikzset{
    shadowed/.style={
        blur shadow={shadow blur steps=5},
        bottom color=Moccasin!30,
        draw=Chocolate!70,
        shade,
        font=\normalfont\Huge\bfseries\scshape,
        rounded corners=8pt,
        top color=SaddleBrown,
    },
    boxes/.style={draw=Chocolate,
        fill=Cornsilk,
        font=\sffamily\small,
        inner sep=5pt,
        rectangle,
        rounded corners=8pt,
        text=Sienna,
    }
}
\newcommand\KLRWTitle{
  \begin{tikzpicture}[remember picture,overlay]
      \node[yshift=-3cm] at (current page.north west)
        {\begin{tikzpicture}[remember picture, overlay]
          \draw[shadowed]
              (30mm,0) rectangle node[white]{KLRW diagrams}
              (\paperwidth-30mm,16mm);
          \node[anchor=west,boxes] at (4cm,0cm) {\@author};
          \node[anchor=east,boxes] at (\paperwidth-4cm,0) {\klrw@version};
         \end{tikzpicture}
        };
   \end{tikzpicture}
   \vspace*{20mm}
}


\def\@oddfoot{\textsc{KLRW diagrams} -- \klrw@version\hfill\thepage}

\usepackage{cite}
\usepackage[colorlinks=true,linkcolor=blue,urlcolor=blue]{hyperref}
\hypersetup{pdfcreator={ Generated by pdfLaTeX },
            pdfinfo={Author  ={ Andrew Mathas },
                     Keywords={ KLRW diagrams,
                                KLR algberas,
                                quiver Hecke algebras,
                                representation theory,
                                algebraic combinatorics
                     },
                     License ={ LaTeX Project Public License v1.3c or later },
                     Subject ={ Typeset diagrams for KLRW algebras },
                     Title   ={ KLRW diagrams - \klrw@version }
            },
}
\makeatother

\begin{document}

  \KLRWTitle

  The KLRW diagram package provides some TikZ styles and \LaTeX\
  commands for drawing KLRW diagrams, following the
  papers~\cite{MathasTubbenhauer:Subdivision,MathasTubbenhauer:BAD,MathasTubbenhauer:FiniteType}.

  \section{Strings in KLRW diagrams}

  The diagrams that generate KLRW algebras contain four types of strings:
  \keyword{red string}
  \keyword{affine},
  \keyword{solid}
  and
  \keyword{ghost}:

  \begin{tikzcode}
    \begin{tikzpicture}
      \draw[red string=red] (1,0)--++(0,2);
      \draw[affine=affine]  (2,0)--++(0,2);
      \draw[solid=solid]    (3,0)--++(0,2);
      \draw[ghost=ghost]    (4,0)--++(0,2);
    \end{tikzpicture}
  \end{tikzcode}

  \textit{Strings should always be drawn from bottom to top}. This is
  because, optionally, you can specify the \keyword{residue} of each of
  these strings, by giving the string a value. For example,
  \keyword{solid=i} sets the residue of this solid string equal to $i$.
  The labels for red, affine and solid strings appear below the strings
  and the labels for ghost strings appear above them. All labels are
  typeset as mathematics in a small font.

  \begin{tikzcode}
    \begin{tikzpicture}
      \draw[red string=0]  (1,0)--++(0,2);
      \draw[affine=\rho_1] (2,0)--++(0,2);
      \draw[solid=i]       (3,0)--++(1,2);
      \draw[solid]         (4,0)--++(-1,2);
      \draw[ghost=j]       (3.9,0)--++(0,2);
    \end{tikzpicture}
  \end{tikzcode}

  As the second solid string shows, the strings do not need to be labelled.

  Depending on the quiver, and the residue of the strings, solid solid
  almost always come with a matching ghost string that is shifted
  \keyword{ghost shift} units to the right. A solid string and its ghost
  can be drawn together by using the \keyword{solid ghost} keyword.

  \begin{tikzcode}
    \begin{tikzpicture}
      \draw[solid ghost]   (0,0)--(1.1,2);
      \draw[solid ghost]   (1.1,0)--(0,2);
      \draw[solid ghost=i] (0.5,0)--(0,1)--(0.5,2);
    \end{tikzpicture}
  \end{tikzcode}

  The (solid and ghost) strings in KLRW diagrams can be decorated with
  finitely man dots, which can be added by using the \keyword{dot} style.
  By default, the dots are placed in the middle of the string but you
  can place the dot at height $h$ on the string, where $h$ is between
  $0$ and $1$, using \keyword{dot=h}:

  \begin{tikzcode}
    \begin{tikzpicture}
      \draw[solid ghost]          (0,0)--(1.1,2);
      \draw[dot=0.2, solid ghost] (1.1,0)--(0,2);
      \draw[dot, solid ghost=i]   (0.5,0)--(0,1)--(0.5,2);
    \end{tikzpicture}
  \end{tikzcode}

  The  \keyword{dot} style can be used to add dots to any of the KLRW
  strings. To add more than one dot to a string use
  \keyword{dot/.list}$=\{...\}$:

  \begin{tikzcode}
    \begin{tikzpicture}
      \draw[solid ghost]          (0,0)--(1.1,2);
      \draw[dot=0.2, solid ghost] (1.1,0)--(0,2);
      \draw[dot, solid ghost=i]   (0.5,0)--(0,1)--(0.5,2);
      \draw[dot/.list={0.2,0.4,0.8}, solid](2.2,0)--++(0,2);
    \end{tikzpicture}
  \end{tikzcode}

  \section{Idempotent diagrams}

    \begin{tikzcode}[sidebyside=false,text above listing]
       \DottedIdempotentA[2,8]{3}{0,1,2,3,4,3,2,1,0}
    \end{tikzcode}

    \begin{tikzcode}[sidebyside=false,text above listing]
        \DottedIdempotentAA[2,8]{3}{0,1,2,3,4,3,2,1,0}
    \end{tikzcode}

    \begin{tikzcode}[sidebyside=false,text above listing]
        \DottedIdempotentC[2,8]{3}{0,1,2,3,4,3,2,1,0}
    \end{tikzcode}

    \begin{tikzcode}[sidebyside=false,text above listing]
        \DottedIdempotentC[2,8]{3}{1,2,3,4,0,3,2,1,0}
    \end{tikzcode}

    \begin{tikzcode}[sidebyside=false,text above listing]
        \DottedIdempotentD[2,8]{3}{0,1,2,3,4,3,2,1,0}
    \end{tikzcode}

    \bibliography{papers}
    \bibliographystyle{andrew}

\end{document}

